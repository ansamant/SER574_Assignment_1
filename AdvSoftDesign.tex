%This is the .tex file created using TeXShop in Mac. This file contains the template Sigpan from Overleaf. 
% SER 574- Advanced Software Design Assignment Number 1. 
% Asurite : svdevi , ansamant

\documentclass[sigplan,screen]{acmart}

\copyrightyear{2019}
\usepackage[compact]{titlesec}
\titlespacing{\section}{1pt}{*1}{*1}
\titlespacing{\subsection}{0.2pt}{*0.00}{*0}
\titlespacing{\subsubsection}{0pt}{*0}{*0}
\def\BibTeX{{\rm B\kern-.05em{\sc i\kern-.025em b}\kern-.08emT\kern-.1667em\lower.7ex\hbox{E}\kern-.125emX}}
\setcopyright{none}

%The start of the Document
\begin{document}

\title{Principles of Distributed Agile Development}
\author{Aditya Samant}
\email{ansamant@asu.edu}
\affiliation{\institution{Arizona State University}}
\author{Sanay Devi}
\email{svdevi@asu.edu}
\affiliation{\institution{Arizona State University}}


% The abstract is a short summary of "what is important when agile teams have to work together with other agile teams to create a software"
\begin{abstract}
Agile teams embrace change, companies today have agile teams running the software development process. Inner-team and cross-team communication is critical in the development of  software. Team effectiveness depends upon trust between the small agile teams not only within the company environment but also between teams across different geographical locations, time zones and cultures \cite{Siva13}. As John Maxwell said "Team work makes the dream work", this same principle applies for agile teams within an organization. The teams should be motivated, aware and informative throughout the software development process. 
\end{abstract}

\maketitle

% Keywords. 
\keywords{teamwork, agile ,team, working, system}


\section{Introduction}
Agile teams are most effective in small projects (<50 people) who have easy access to user, business experts and are developing non-safety critical projects \cite{Dingsoyr} .However, to meet requirements for large scale projects a globalized approach with outsourced teams is becoming common \cite{Jeff},where different teams must work together cohesively, flexibly and rapidly to deliver  a product that conforms to requirements specifications. In order for small team based agile practices to be effective on  a larger scale, careful consideration must be given to the collaboration between teams and the overall architecture. Section 2 is entirely dedicated to highlight the various facets in which teamwork affects large scale agile projects. Subsection 2.1 informs the reader which scrum approach of the three scrum approaches \cite{Jeff} is the best to use in a large scale scrum project, subsection 2.2 highlights the challenges in inter team communications over large scale projects and subsection 2.3 highlights how architecture and design affects teams in large projects.
 
\section{Team Work}
Most companies today consist of project teams ranging from two to several hundred people. These teams are further split into groups and each group is then responsible for developing a part of the system. The teams can be grouped into three categories. Isolated Scrum Teams, Distributed Scrum Of Scrums and Totally Integrated Scrums \cite{Jeff}. 
The isolated teams are usually on site teams and communication between them is usually by in-person meetings and "elevator meetings".The main communication problem occurs when the teams are distributed overseas since teams will portray differences in work styles, in the worst case outsourced teams may not use Scrum and be productive using the waterfall approach. This issue needs to be addressed, and all teams should in fact use the same methodology and stick to it.\\
To form a group of professional individuals, the critical management task is to find a right balance of technical skills, experience and personalities. However, for a team to be actually productive and successful, the team should be cohesive and have team spirit \cite{Somerville}. Each team member should motivate others and  be loyal to each other. When problems arise or when sudden changes are required, the group as a whole and not individually should be able to adapt to the changes and overcome the problems.\\
The idea of teamwork encapsulates a set of values that encourage listening and responding constructively to views expressed by others, giving others the benefit of the doubt, providing support, and recognizing the interests and achievements of others \cite{Moe}.
The subsequent subsections highlight what is important when Agile Teams have to work together. 
\begin {comment}
Block comment for the sake of removing this section and expanding the next one.
\subsection{Selecting Team Members}
The main responsibility of the manager of a project is to create a cohesive group of teams and organize them so they can work together effectively. Due to budget constraints and lack of specific skills, companies outsource teams from different regions and managers do not have the complete say in the process of team selection. The problem arises when individuals are motivated by their own work and who have their own ideas about how technical problems should be solved. The manager should select teams and their members based on having similar complementary personalities and not rely solely on technical capabilities. Individuals who are interaction-oriented help facilitate communication within the team and are the key in holding the sanity of the entire team together \cite{Somerville}. \\
When building a team, each participant's potential contribution to the process has to be evaluated.The ideal team member is a part of the team and identifies himself/herself with the team's goals, represents colleagues and has the time to participate through the entire process \cite{Gautam17}.\\
The key factors to select the team members are education,\\
training, application domain, technology experience, communication ability, adaptability and problem solving skills \cite{Somerville}.  
\end{comment}

\subsection{Scrum Approach}
Jeff Sutherland et al. in 2007, conducted a case study and provided three different approaches to conduct Scrum in large scaled processes of which Isolated Scrum methodology was considered the least efficient since the teams had no mechanism for regular communications, the other methodologies were viewed favorably by Sutherland and folk and largely that was due to the presence of regularly scheduled Scrum of Scrums meetings, where a Chief Scrum Master and Chief Product Owner will discuss the overall shape of the project with the individual team leads \cite{Jeff}. The scrum of scrums meeting greatly increases the efficiency of the project and is considered by the authors to be the second best scrum approach for distributed development.However, it is still a bit inefficient when trying to create a proper architecture that is workable with parallel teamwork \cite{Dingsoyr}.The best approach for distributed development which further increases communications between the teams is by dispersing team members through different sites. By doing so team members visiting the sites are exposed to a different culture and work environment and thus get a better understanding of the project as a whole.

\subsection{Team Communication}
Inter-team communication is one the six issues in distributed development \cite{Jeff}, since a failure to communicate properly with teammates will lead to production delays and conflicts in code implementation. Traditional agile methodology emphasizes self-management, in a small team a team member is cross-functional and responsible for every phase in the agile development of the project \cite{Dingsoyr}. Due to the eclectic nature of modern large-scale projects the teams can be spread across time zones, cultures and languages. Many a times to make a project within the financial, quality and time constraints provided by the client, companies will outsource portions of the project to other companies.  \\
Bjornson et al. studied multiple large scale agile projects and determined that a shared mental model, closed-looped communication and trust are the essential mechanisms through which a project can succeed \cite{Bjornson}. \\
These mechanisms are interrelated and as such without effective communication mechanisms the building of a shared mental model and the fostering trust within teams is less likely to take place. Dingsoyr and Moe in 2014 determined that one way to foster a strong communication and understanding between different teams is by conducting inter team workshops, the goal of the workshop is to understand each other's work  culture and to develop a shared team language.\\
 Another key means through which a shared team language and cohesion can increase is having the individuals have a "knowledge network" to be aware of fields and area beyond the scope of their team which will aide in the self-management of the individual teammates\cite{Dingsoyr}.\\

\subsection{Agile and Architecture}
Architectural design can easily become problematic in large scale agile projects due to multiple different teams working in parallel on different sections of the product \cite{Dingsoyr}. Many agile teams favor code refactoring instead of sound architectural design decrying it as big architecture upfront \cite{Nord}. However, in large scale systems refactoring is a poor substitute for sound software and system architecture due to the frequent changes agile is prone to make throughout its lifecycle. Conversely, good architectural design enhances agility in large projects. A large-scale agile process with an early architecture focus defines the implementation structure that divides the organization into small teams focusing on either the infrastructure or on the independent features themselves\cite{Nord}. A good software and system architecture also allows for the creation of a common vocabulary and common culture, a guide for release planning and configuration management and many other benefits \cite{Nord}.
Architects focus on three separate concerns for agile software development: Architecture of the system, the structure of the development and production infrastructure \cite{Dingsoyr}. For agility to remain over time in a large scale project these three concerns need to be kept aligned over the course of time. Nord et al maintains that architectural requirements must be drawn along with functional and non-functional requirements with respect to changing needs. Architectural tactics such as vertical and horizontal decomposition (which deal with task distribution), matrix and augmented team structures, architectural runways (for better maintainability) and deployability tactics provide a healthy framework to maintain the alignment of the three concerns \cite{Nord}. 
 
 \section{Conclusion}
 Individuals learn more and  learn better working together rather than in isolation or competing against each other. Being creative and innovative are important factors for teamwork along with having a common vocabulary and work culture. When the teams together can solve problems quickly and effectively, the entire development process benefits from it. The performance of teams is also dependent on how and where each team is placed in the organization. Proper placement of teams is driven by good system and software architectural designs. Being encouraged and accepted within the organization will motivate the individual in teams to perform better and lead to more creative innovation. Therefore, a collaborative work environment with the correct organizational context, is the base on which companies are built, it enhances the team performance and caters to the overall success of a company.

\bibliographystyle{ACM-Reference-Format}
%Please change location of the .bib file here, wherever it gets saved on your local machine. 
\bibliography{/Users/sanaydevi/Desktop/SER574Assignment11/AdvSoftDesign.bib}

\end{document}
